%%%%%%%%%%%%%%%%%%%%%%%%%%%%%%%%%%%%%%%%%%%%%%%%%%
% When making a high-res pdf for printing, use
% the following define. Changes resolution and colour
% of hyperlinks.

\def\printable{true}

%%%%%%%%%%%%%%%%%%%%%%%%%%%%%%%%%%%%%%%%%%%%%%%%%%
% If the following define is commented,
% not all packages will be used, resulting in
% a faster compile.

\def\generatefull{true}

%%%%%%%%%%%%%%%%%%%%%%%%%%%%%%%%%%%%%%%%%%%%%%%%%%
% Mandatory packages

% Language
\usepackage[english,swedish]{babel}

% Modern font encoding
\usepackage[utf8]{inputenc}

% Really good for quoting text's, source code, etc
\usepackage{listings}
% For multiple rows within tables
\usepackage{multirow}
% For good-looking figures side by side
\usepackage{subfig}

\usepackage{units}

% Nice math output
\usepackage{amsmath}

\usepackage{MnSymbol}

\usepackage{ctable}

% Custom toc-depth for different parts of the document
\usepackage{tocvsec2}

%\usepackage{txfonts} % Screws up all the equations, using upgreek for straight \mu:s instead
\usepackage{upgreek}

% Packages that are not crucial may be put in here to speed things up.
% (Ie, packages dealing only with aesthetics
\ifdefined\generatefull
	% Not sure what this does
	\usepackage{ae}
	% Adjusts textblocks in subtle ways (looks better)
	\usepackage{microtype}		
	% Included in the cls-file or not 'yet' needed
	%\usepackage{lipsum}
	%\usepackage[T1]{fontenc}


	% Nice captions
	\usepackage{caption}
	\captionsetup{margin=10pt,font=small,labelfont=bf}
	\captionsetup[table]{position=top}

\fi
%%%%%%%%%%%%%%%%%%%%%%%%%%%%%%%%%%%%%%%%%%%%%%%%%%
% LaTeXDraw
%\usepackage[usenames,dvipsnames]{pstricks}
%\usepackage{epsfig}
%\usepackage{pst-grad} % For gradients
%\usepackage{pst-plot} % For axes
% Doesn't work. Using .eps instead.


%%%%%%%%%%%%%%%%%%%%%%%%%%%%%%%%%%%%%%%%%%%%%%%%%%
%	pdflatex setup
% 	Use with:
%		pdflatex --shell-escape filename

%\usepackage[pdftex]{color,graphicx} % Already in class-file for logo
\usepackage{epstopdf}

%%%%%%%%%%%%%%%%%%%%%%%%%%%%%%%%%%%%%%%%%%%%%%%%%%


%% Check if we are building for the web or for printing
\ifdefined\printable
	\def \backref {false}
	\usepackage[res300]{optional}
\else
	% Printable was not defined, therefore:
	%	Set settings according to the web
	\def \backref {true}
	\usepackage[res72]{optional}
\fi

%%%%%%%%%%%%%%%%%%%%%%%%%%%%%%%%%%%%%%%%%%%%%%%%%%
% Biblatex will soon be the standard citations-package
\usepackage[
%		sorting=nyt, %previous
		sorting=none,
		hyperref=auto,
	%	style=authoryear-icomp-tt, % Special style to avoid broken authoryear
%		style=authoryear, %previous
		style=numeric-comp,
%		citestyle=authoryear,
		doi=false,
		isbn=false,
		bibstyle=numeric,
%		autocite=footnote, %previous
		autocite=plain,
		backref=\backref
	]{biblatex}

\bibliography{references}


%%%%%%%%%%%%%%%%%%%%%%%%%%%%%%%%%%%%%%%%%%%%%%%%%%
% Hyperref
\usepackage[pdfauthor={Richard Abrahamsson, Anders Bennehag},%
pdftitle={S-band Multi-function Down-converter GaAs MMIC},%
pdftex]{hyperref}
% Hypcap makes hyperlinks really point to the picture,
% not to the caption below
\usepackage[all]{hypcap} 

\definecolor{black}{rgb}{0.0,0.0,0.0}
\definecolor{darkblue}{rgb}{0.0,0.0,0.5}
\ifdefined\printable
	% Make links black (better for printing)
	\hypersetup{pdftex,colorlinks,breaklinks,
		linkcolor=black,urlcolor=black,
		anchorcolor=black,citecolor=black}
\else
	% Printable was not defined, therefore: Set settings according to the web
	\hypersetup{pdftex,colorlinks,breaklinks,
		linkcolor=darkblue,urlcolor=darkblue,
		anchorcolor=darkblue,citecolor=darkblue}
\fi


%%%%%%%%%%%%%%%%%%%%%%%%%%%%%%%%%%%%%%%%%%%%%%%%%%
%Custom commands
\newcommand{\includerect}[2]{\includegraphics[trim=50 60 40 60, clip, width=#1]{#2}}
\newcommand{\includesmith}[2]{\includegraphics[trim=60 40 60 50, clip, width=#1]{#2}}
\newcommand{\mum}{$\upmu$m}
\newcommand{\disclaimer}{ Temperature at \unit[25]{$^\circ$C} and no parameter spread applied.}
\newcommand{\scalemum}{ Dimensions in \mum.}
\newcommand{\nf}{\emph{NF}}

\usepackage{ifthen}
 
\newcommand{\newevenside}{
	\ifthenelse{\isodd{\thepage}}{\newpage}{
	\newpage
        \phantom{placeholder} % doesn't appear on page
	\thispagestyle{empty} % if want no header/footer
	\newpage
	}
}